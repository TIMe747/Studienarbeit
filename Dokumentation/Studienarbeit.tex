%Art des Dokuments%
\documentclass[a4paper, 11pt]{article}
%Packete
\usepackage{ifxetex,ifluatex}
\usepackage{etoolbox}
\usepackage[svgnames]{xcolor}
\usepackage{amssymb}
\usepackage{tikz}
\usepackage{tcolorbox}
\usepackage{framed}

% conditional for xetex or luatex
\newif\ifxetexorluatex
\ifxetex
\xetexorluatextrue
\else
\ifluatex
\xetexorluatextrue
\else
\xetexorluatexfalse
\fi
\fi
\ifxetexorluatex
\usepackage{fontspec}
\usepackage{libertine}
\newfontfamily\quotefont[Ligatures=TeX]{Linux Libertine O}
\else
\usepackage[utf8]{inputenc}
\usepackage[T1]{fontenc}
\usepackage{libertine} 
\newcommand*\quotefont{\fontfamily{LinuxLibertineT-LF}}
\fi

\newcommand*\quotesize{60}
\newcommand*{\openquote}
{\tikz[remember picture,overlay,xshift=-4ex,yshift=-2.5ex]
	\node (OQ) {\quotefont\fontsize{\quotesize}{\quotesize}\selectfont``};\kern0pt}

\newcommand*{\closequote}[1]
{\tikz[remember picture,overlay,xshift=4ex,yshift={#1}]
	\node (CQ) {\quotefont\fontsize{\quotesize}{\quotesize}\selectfont''};}

\colorlet{shadecolor}{Azure}

\newcommand*\shadedauthorformat{\emph}
\newcommand*\authoralign[1]{%
	\if#1l
	\def\authorfill{}\def\quotefill{\hfill}
	\else
	\if#1r
	\def\authorfill{\hfill}\def\quotefill{}
	\else
	\if#1c
	\gdef\authorfill{\hfill}\def\quotefill{\hfill}
	\else\typeout{Invalid option}
	\fi
	\fi
	\fi}

\newenvironment{shadequote}[2][l]%
{\authoralign{#1}
	\ifblank{#2}
	{\def\shadequoteauthor{}\def\yshift{-2ex}\def\quotefill{\hfill}}
	{\def\shadequoteauthor{\par\authorfill\shadedauthorformat{#2}}\def\yshift{2ex}}
	\begin{snugshade}\begin{quote}\openquote}
		{\shadequoteauthor\quotefill\closequote{\yshift}\end{quote}\end{snugshade}}


\usepackage[ngerman]{babel}
\usepackage{color}
\usepackage[a4paper, lmargin={4cm}, rmargin={2cm}, tmargin={2,5cm}, bmargin={2,5cm}]{geometry}
\usepackage{graphicx}
\usepackage{setspace}
\usepackage{framed}
\usepackage{url}
\usepackage{eurosym}
\usepackage{acronym}
\usepackage{listings}
\usepackage{color}
\usepackage{longtable}
\usepackage{courier}
\usepackage[multiple]{footmisc}
\usepackage{selinput}
\usepackage{array}
\usepackage{multirow}
\usepackage{longtable}
\usepackage{booktabs}
\usepackage[framemethod=TikZ]{mdframed}

\newenvironment{theo}[2][]{%
	\refstepcounter{theo}
	
	
	\begin{mdframed}[]\relax}{%
\end{mdframed}}
\ifstrempty{#1}%

{\mdfsetup{%
		frametitle={%
			\tikz[baseline=(current bounding box.east),outer sep=0pt]
			\node[anchor=east,rectangle,fill=blue!20]
			{\strut Definition~\thetheo};}
	}%

}{\mdfsetup{%
		frametitle={%
			\tikz[baseline=(current bounding box.east),outer sep=0pt]
			\node[anchor=east,rectangle,fill=blue!20]
			{\strut Definition~\thetheo};}%
	}%
}%

\mdfsetup{%
	innertopmargin=10pt,linecolor=blue!20,%
	linewidth=2pt,topline=true,%
	frametitleaboveskip=\dimexpr-\ht\strutbox\relax%
}


\newcommand*{\theadtext}[1]{{\tiny #1}}
\newcommand*{\thead}[1]{\multicolumn1{@{}c@{}}{\theadtext{#1}}}
\newcolumntype{C}[1]{>{\centering\arraybackslash\hspace{0pt}}p{#1}}
\newcolumntype{L}[1]{>{\raggedright\arraybackslash\hspace{0pt}}p{#1}}
\newcolumntype{R}[1]{>{\raggedleft\arraybackslash\hspace{0pt}}p{#1}}
\newcounter{pos}
\newcommand*{\pos}{\refstepcounter{pos}\thepos}
\newcommand{\sectionnumbering}[1]{% 
	\setcounter{section}{0}% 
	\renewcommand{\thesection}{\csname #1\endcsname{section}}} 
\newcommand{\Autor}{Tim Saupp}
\newcommand{\MatrikelNummer}{2742603}
\newcommand{\Kursbezeichnung}{TINF15B3}

\newcommand{\Was}{Studienarbeit}
\newcommand{\Titel}{Titel ausstehend}
\newcommand{\AbgabeDatum}{18.09.2017}
\newcommand{\Dauer}{03.07.2017-15.09.2017}
\newcommand{\Abschluss}{Bachelor of Engineering}
\newcommand{\Studiengang}{Informationstechnik}
\newcommand{\BetreuerDHBW}{Prof. Dr. Lausen}

\makeatletter
\newcommand*{\maintoc}{% Hauptinhaltsverzeichnis
	\begingroup
	\@fileswfalse% kein neues Verzeichnis öffnen
	\renewcommand*{\appendixattoc}{% Trennanweisung im Inhaltsverzeichnis
		\value{tocdepth}=-10000 % lokal tocdepth auf sehr kleinen Wert setzen
	}%
	\tableofcontents% Verzeichnis ausgeben
	\endgroup
}
\newcommand*{\appendixtoc}{% Anhangsinhaltsverzeichnis
	\begingroup
	\edef\@alltocdepth{\the\value{tocdepth}}% tocdepth merken
	\setcounter{tocdepth}{-10000}% Keine Verzeichniseinträge
	\renewcommand*{\contentsname}{% Verzeichnisname ändern
		Verzeichnis der Anh\"ange}%
	\renewcommand*{\appendixattoc}{% Trennanweisung im Inhaltsverzeichnis
		\setcounter{tocdepth}{\@alltocdepth}% tocdepth wiederherstellen
	}%
	\tableofcontents% Verzeichnis ausgeben
	\setcounter{tocdepth}{\@alltocdepth}% tocdepth wiederherstellen
	\endgroup
}
\newcommand*{\appendixattoc}{% Trennanweisung im Inhaltsverzeichnis
}
\g@addto@macro\appendix{% \appendix erweitern
	\if@openright\cleardoublepage\else\clearpage\fi% Neue Seite
	\addtocontents{toc}{\protect\appendixattoc}% Trennanweisung in die toc-Datei
}
\makeatother
%Grafische Einstellungen%
\clubpenalty = 100
\widowpenalty = 100
\definecolor{dkgreen}{rgb}{0,0.6,0}
\definecolor{gray}{rgb}{0.5,0.5,0.5}
\definecolor{mauve}{rgb}{0.58,0,0.82}
\definecolor{eggshell}{rgb}{0.94, 0.92, 0.84}
\lstset{
	backgroundcolor=\color{eggshell},
	basicstyle=\footnotesize\ttfamily,
	frame=single,
	breaklines=true,	
	commentstyle=\color{dkgreen},
	captionpos=b,
	keywordstyle=\color{blue},
	stringstyle=\color{mauve},
	tabsize=2,
	language=Java,
	numbers=left,
	numbersep=5pt,
	numberstyle=\tiny\color{gray},
	showstringspaces=false}
%Umbenennung der Zusammenfassung zu Abstract%
\addto\captionsngerman{\renewcommand{\abstractname}{Abstract}}

%%%%%%%%%%%%%%%%%%%%%%%%%%%%%%%%%%%%%%%%%%%%%%%%%%%%%
% Tim Saupp Studienarbeit 2017-2018 5.& 6. Semester %
%%%%%%%%%%%%%%%%%%%%%%%%%%%%%%%%%%%%%%%%%%%%%%%%%%%%%
%Anfang des Dokuments%
\begin{document}
\begin{titlepage}
	\begin{center}
		\vspace*{-2cm}
		\hfill\includegraphics[width=5cm]{dhbw-logo}\\[2cm]
		{\Huge \Titel}\\[2cm]
		{\Huge\scshape \Was}\\[2cm]
		{\large für die Prüfung zum}\\[0.5cm]
		{\Large \Abschluss}\\[0.5cm]
		{\large des Studienganges \Studiengang}\\[0.5cm]
		{\large an der}\\[0.5cm]
		{\large Dualen Hochschule Baden-Württemberg Karlsruhe}\\[0.5cm]
		{\large von}\\[0.5cm]
		{\large\bfseries \Autor}\\[1cm]
		{\large Abgabedatum \AbgabeDatum}
		\vfill
	\end{center}
	\begin{tabular}{l@{\hspace{2cm}}l}
		Bearbeitungszeitraum	         & \Dauer 			\\
		Matrikelnummer	                 & \MatrikelNummer		\\
		Kurs			         & \Kursbezeichnung		\\
		Gutachter der Studienakademie	 & \BetreuerDHBW		\\
	\end{tabular}
\end{titlepage}
\newpage
\thispagestyle{empty}
\begin{framed}
	\begin{center}
		\Large\bfseries Erklärung
	\end{center}
	\medskip
	\noindent
	Ich versichere hiermit, dass ich meine \Was\ mit
	dem Titel: {\Titel} selbstständig verfasst und keine anderen als die angegebenen Quellen und
	Hilfsmittel benutzt habe. Ich versichere zudem, dass die eingereichte elektronische Fassung mit der
	gedruckten Fassung übereinstimmt.
	
	\vspace{3cm}
	\noindent
	\underline{\hspace{4cm}}\hfill\underline{\hspace{6cm}}\\
	Ort~~~~~Datum\hfill Unterschrift\hspace{4cm}
\end{framed}
\newpage
\begin{framed}
	\begin{center}
		\Large\bfseries Sperrvermerk
	\end{center}
	\medskip
	\noindent
	Der Inhalt dieser Arbeit darf weder als Ganzes noch in Auszügen Personen
	außerhalb des Prüfungsprozesses und des Evaluationsverfahrens zugänglich gemacht
	werden, sofern keine anders lautende Genehmigung der Ausbildungsstätte vorliegt.
\end{framed}
\newpage
\pagenumbering{Roman} 
\begin{abstract}
Hier Abstract.
\end{abstract}
\newpage
\maintoc           % Inhaltsverzeichnis hier ausgeben%
\newpage
\listoffigures             % Liste der Abbildungen%
\newpage
\listoftables              % Liste der Tabellen%
\newpage
\section*{\Large \textbf Abkürzungsverzeichnis}  
\begin{acronym}[Bash]
	\acro{DUMMY}{DUMMY}
\end{acronym}
\newpage
\pagenumbering{arabic} 
\section{Projektbeschreibung}
\subsection{Motivation}
\subsection{Ziel der Arbeit}
\subsection{Kapitelübersicht}
\section{Grundlagen \& Begriffe}
\subsection{Emergente Superorgansimen}
\subsection{Metaheuristik}
\subsection{Symmetrisch verteilte Algorithmen}
\subsection{Agentenbasierte Modellierung}
\subsection{Schwarmintelligente Algorithmen}
\subsubsection{Particle Swarm Optimization}
\subsubsection{Ant Colony Optimization}
\subsubsection{Bee Colony Optimization}
\subsection{Optimierungsproblem}
\section{Anforderungsanalyse}
\section{Entwurf \& Design}
\subsection{Softwarearchitektur}
\subsubsection{Model View Controller}
\subsubsection{Klassendiagramm}
\section{Implementierung}
\subsection{Simulator}
\subsubsection{Particle Swarm Optimization}
\subsubsection{Ant Colony Optimization}
\subsubsection{Bee Colony Optimization}
\section{Fazit}
\section{Ausblick}





\newpage
\begin{thebibliography}{sotief}
	\bibitem{1}DUMMY
\end{thebibliography}
\newpage
\appendix
\appendixtoc
\newpage
\section{Anhang 1} 
\subsection{Software Requirements Specification}
\subsection{Quellcode}
\end{document}


